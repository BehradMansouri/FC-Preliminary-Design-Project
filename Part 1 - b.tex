%LaTeX reported generated by EES
\documentclass[10pt,fleqn]{article}
\mathindent 0.0in
\usepackage[ansinew]{inputenc}
\usepackage{times}
\usepackage{graphicx}
\usepackage{color}
\usepackage{textcomp}
\definecolor{silver}{rgb}{0.75,0.75,0.75}
\definecolor{gray}{rgb}{0.5,0.5,0.5}
\definecolor{aqua}{rgb}{0.5,1,1}
\definecolor{navy}{rgb}{0.0,0.0,0.5}
\definecolor{orange}{rgb}{1.0,0.5,0.0}
\definecolor{teal}{rgb}{0.25,0.5,0.5}
\definecolor{olive}{rgb}{0.5,0.5,0.0}
\definecolor{purple}{rgb}{0.5,0.0,0.5}
\definecolor{brown}{rgb}{0.5,0.25,0.0}
\definecolor{fuchsia}{rgb}{1.0,0.5,1.0}
\definecolor{buff}{rgb}{1.0,0.94,0.80}
\definecolor{lime}{rgb}{0.5,1.0,0.0}
\setlength{\headsep}{-0.6in}
\setlength{\textheight}{9in}
\setlength{\footskip}{1.0 in}
\setlength{\oddsidemargin}{-0.2in}
\setlength{\evensidemargin}{-0.2in}
\setlength{\textwidth}{6.8in}
\usepackage{longtable}
\def\headline#1{\hbox to \hsize{\hrulefill\quad\lower .3em\hbox{#1}\quad\hrulefill}}
\newcommand{\abs}[1]{\left|#1\right|}
\newcommand{\F}[1]{\mbox{$#1$}}
\newcommand{\K}[1]{\mbox{\sf#1\ \ \mit}}
\newcommand{\KS}[1]{\mbox{\sf\ \ #1\ \ \mit}}
\newcommand{\SC}[1]{\mbox{`#1'}\  }
\newcommand{\V}[1]{\mbox{$ #1 $}}
\newcommand{\I}{\mbox{\hspace{0.20in}}}
\newcommand{\temperature}{\mathrm{T}}
\newcommand{\pressure}{\mathrm{P}}
\newcommand{\volume}{\mathrm{v}}
\newcommand{\density}{\mathrm{\rho}}
\newcommand{\intenergy}{\mathrm{u}}
\newcommand{\enthalpy}{\mathrm{h}}
\newcommand{\entropy}{\mathrm{s}}
\newcommand{\molarmass}{\mathrm{MW}}
\newcommand{\enthalpyfusion}{\mathrm{\Delta h_{fusion}}}
\newcommand{\quality}{\mathrm{x}}
\newcommand{\viscosity}{\mathrm{\mu}}
\newcommand{\conductivity}{\mathrm{k}}
\newcommand{\prandtl}{\mathrm{P_r}}
\newcommand{\cp}{\mathrm{c_p}}
\newcommand{\cv}{\mathrm{c_v}}
\newcommand{\specheat}{\mathrm{c_p}}
\newcommand{\soundspeed}{\mathrm{c}}
\newcommand{\wetbulb}{\mathrm{wb}}
\newcommand{\humrat}{\mathrm{\omega}}
\newcommand{\acentricfactor}{\mathrm{\omega}}
\newcommand{\relhum}{\mathrm{\phi}}
\newcommand{\dewpoint}{\mathrm{DP}}
\newcommand{\volexpcoef}{\mathrm{\beta}}
\newcommand{\compressibilityfactor}{\mathrm{Z}}
\newcommand{\surfacetension}{\mathrm{\gamma}}
\newcommand{\tcrit}{\mathrm{T_{crit}}}
\newcommand{\pcrit}{\mathrm{P_{crit}}}
\newcommand{\vcrit}{\mathrm{v_{crit}}}
\newcommand{\ttriple}{\mathrm{T_{triple}}}
\newcommand{\fugacity}{\mathrm{fugacity}}
\newcommand{\tsat}{\mathrm{T_{sat}}}
\newcommand{\psat}{\mathrm{P_{sat}}}
\newcommand{\eklj}{\mathrm{ek_{LJ}}}
\newcommand{\sigmalj}{\mathrm{\sigma_{LJ}}}
\newcommand{\isentropicexponent}{\mathrm{k_{s}}}
\newcommand{\thermaldiffusivity}{\mathrm{\alpha}}
\newcommand{\kinematicviscosity}{\mathrm{\nu}}
\newcommand{\isothermalcompress}{\mathrm{K_{T}}}
\newcommand{\henryconstantwater}{\mathrm{HenryConst}}
\begin{document}
\begin{center}
\bf \mbox{PART 1 - B}
\vspace{0.2 in}
\end{center}
\subsection*{Equations}
\begin{equation}
\label{EES Eqn:1}
\K{function} \F{getenthalpy}{ \left( S\$,\ T \right) } 
\end{equation}

\vspace{0.04in}
\noindent
\rm This function uses the NASA external procedure to return the enthalpy of reactant species S\$ at T.
\begin{equation}
\I \label{EES Eqn:2}
\K{call} \F{NASA}{ \left( S\$,\ T:\V{cp} ,\ \V{geTenthalpy} ,\ s \right) } 
\end{equation}
\begin{equation}
\label{EES Eqn:3}
\K{end} \V{getenthalpy}  
\end{equation}
\vspace{0.1 in}
\begin{equation}
\label{EES Eqn:4}
\K{procedure} \F{enthalpygibbs}{ \left( S\$,\ T:h,\ g \right) } 
\end{equation}

\vspace{0.04in}
\noindent
\rm This procedure uses the NASA external procedure to return the enthalpy and Gibbs energy of product species S\$ at T.
\begin{equation}
\I \label{EES Eqn:5}
\K{call} \F{NASA}{ \left( S\$,\ T:\V{cp} ,\ h,\ s \right) } 
\end{equation}
\begin{equation}
\I \label{EES Eqn:6}
g:=h-T\cdot s 
\end{equation}
\begin{equation}
\label{EES Eqn:7}
\K{end} h_{g} 
\end{equation}

\vspace{0.04in}
\noindent
\vspace{0.1 in}
\rm Reaction:   0.9 CH4  + 0.1 C2H6 +  a (O2 + 3.76 N2)   $\leftarrow$---$\rightarrow$  b CO2  +  c H2O  +  d N2  +  e O2  +  f O  +  g N  +  j H  +  k OH   +  m H2  +  n  CO +  $>$$>$$>$p CH4  +  q C2H6$<$$<$$<$
\begin{equation}
\label{EES Eqn:8}
T_{air} = 460   \   \left[ \rm K \right] 
\end{equation}
\rm
\begin{equation}
\label{EES Eqn:9}
T_{fuel} = 430   \   \left[ \rm K \right] 
\end{equation}
\rm
\begin{equation}
\label{EES Eqn:10}
T_{final,1}= 1700   \   \left[ \rm K \right] 
\end{equation}
\rm
\begin{equation}
\label{EES Eqn:11}
T_{final,2}= 1600   \   \left[ \rm K \right] 
\end{equation}
\rm
\begin{equation}
\label{EES Eqn:12}
T_{final,3}= 1300   \   \left[ \rm K \right] 
\end{equation}
\rm
\begin{equation}
\label{EES Eqn:13}
P = 1000   \   \left[ \rm kPa \right] 
\end{equation}
\rm
\begin{equation}
\label{EES Eqn:14}
P_{ref} = 100   \   \left[ \rm kPa \right] 
\end{equation}
\rm
\begin{equation}
\label{EES Eqn:15}
R = R\#  	 
\mbox{\I Universal gas constant}
\end{equation}
\begin{equation}
\label{EES Eqn:16}
h_{N2,inlet} = \F{getenthalpy}{ \left( \SC{N2},\ T_{air} \right) } 
\end{equation}
\begin{equation}
\label{EES Eqn:17}
h_{O2,inlet} = \F{getenthalpy}{ \left( \SC{O2},\ T_{air} \right) } 
\end{equation}
\begin{equation}
\label{EES Eqn:18}
h_{CH4,inlet} = \F{getenthalpy}{ \left( \SC{CH4},\ T_{fuel} \right) } 
\end{equation}
\begin{equation}
\label{EES Eqn:19}
h_{C2H6,inlet} = \F{getenthalpy}{ \left( \SC{C2H6},\ T_{fuel} \right) } 
\end{equation}
\begin{equation}
\label{EES Eqn:20}
h_{fuel,inlet} = 0.9 \cdot  h_{CH4,inlet} + 0.1 \cdot  h_{C2H6,inlet} 
\end{equation}
\begin{equation}
\label{EES Eqn:21}
\K{duplicate} i=1,\ 3 
\end{equation}

\vspace{0.04in}
\noindent
\rm Stoichiometry for a basis of 1 kmol of fuel
\begin{equation}
\I \label{EES Eqn:22}
1.1 = b_{i} + n_{i}	 
\mbox{\I Carbon balance}
\end{equation}
\begin{equation}
\I \label{EES Eqn:23}
4.2 =  2 \cdot  c_{i} + j_{i} + k_{i} + 2 \cdot  m_{i}	 
\mbox{\I Hydrogen balance}
\end{equation}
\begin{equation}
\I \label{EES Eqn:24}
2 \cdot  a_{i} = 2 \cdot  b_{i} + c_{i} + 2 \cdot  e_{i} + f_{i} + k_{i} + n_{i}	 
\mbox{\I Oxygen balance}
\end{equation}
\begin{equation}
\I \label{EES Eqn:25}
3.76 \cdot  2 \cdot  a_{i} = 2 \cdot  d_{i} + g_{i}	 
\mbox{\I Nitrogen balance}
\end{equation}

\vspace{0.04in}
\noindent
\rm Total moles of gas and mole fractions.
\begin{equation}
\I \label{EES Eqn:26}
n_{tot,i} =  \left( b_{i} +c_{i} +d_{i} +e_{i} +f_{i} +g_{i} + j_{i} + k_{i} + m_{i} + n_{i} \right)  
\end{equation}
\begin{equation}
\I \label{EES Eqn:27}
y_{CO2,i} \cdot  n_{tot,i} = b_{i} 
\end{equation}
\begin{equation}
\I \label{EES Eqn:28}
y_{H2O,i} \cdot  n_{tot,i} = c_{i} 
\end{equation}
\begin{equation}
\I \label{EES Eqn:29}
y_{N2,i} \cdot  n_{tot,i} = d_{i} 
\end{equation}
\begin{equation}
\I \label{EES Eqn:30}
y_{O2,i} \cdot  n_{tot,i} = e_{i} 
\end{equation}
\begin{equation}
\I \label{EES Eqn:31}
y_{O,i} \cdot  n_{tot,i} = f_{i} 
\end{equation}
\begin{equation}
\I \label{EES Eqn:32}
y_{N,i} \cdot  n_{tot,i} = g_{i} 
\end{equation}
\begin{equation}
\I \label{EES Eqn:33}
y_{H,i} \cdot  n_{tot,i} = j_{i} 
\end{equation}
\begin{equation}
\I \label{EES Eqn:34}
y_{OH,i} \cdot  n_{tot,i} = k_{i} 
\end{equation}
\begin{equation}
\I \label{EES Eqn:35}
y_{H2,i} \cdot  n_{tot,i} = m_{i} 
\end{equation}
\begin{equation}
\I \label{EES Eqn:36}
y_{CO,i} \cdot  n_{tot,i} = n_{i} 
\end{equation}

\vspace{0.04in}
\noindent
\rm The following equations provide the enthalpy for each chemical species at the inlet tempretures and T$_{final}$ and the reference pressure of 10 bar. The NASA external procedure is used in 	the Function getenthalpy to calculate h at the equilibrium temperature, which is determined from an energy balance.
\begin{equation}
\I \label{EES Eqn:37}
\K{call} \F{enthalpygibbs}{ \left( \SC{CO2},\ T_{final,i} : h_{CO2,i} ,\ g^{o}_{CO2,i} \right) } 
\end{equation}
\begin{equation}
\I \label{EES Eqn:38}
\K{call} \F{enthalpygibbs}{ \left( \SC{H2O},\ T_{final,i} : h_{H2O,i} ,\ g^{o}_{H2O,i} \right) } 
\end{equation}
\begin{equation}
\I \label{EES Eqn:39}
\K{call} \F{enthalpygibbs}{ \left( \SC{N2},\ T_{final,i} : h_{N2,i} ,\ g^{o}_{N2,i} \right) } 
\end{equation}
\begin{equation}
\I \label{EES Eqn:40}
\K{call} \F{enthalpygibbs}{ \left( \SC{O2},\ T_{final,i} : h_{O2,i} ,\ g^{o}_{O2,i} \right) } 
\end{equation}
\begin{equation}
\I \label{EES Eqn:41}
\K{call} \F{enthalpygibbs}{ \left( \SC{O},\ T_{final,i} : h_{O,i} ,\ g^{o}_{O,i} \right) } 
\end{equation}
\begin{equation}
\I \label{EES Eqn:42}
\K{call} \F{enthalpygibbs}{ \left( \SC{N},\ T_{final,i} : h_{N,i} ,\ g^{o}_{N,i} \right) } 
\end{equation}
\begin{equation}
\I \label{EES Eqn:43}
\K{call} \F{enthalpygibbs}{ \left( \SC{H},\ T_{final,i} : h_{H,i} ,\ g^{o}_{H,i} \right) } 
\end{equation}
\begin{equation}
\I \label{EES Eqn:44}
\K{call} \F{enthalpygibbs}{ \left( \SC{OH},\ T_{final,i} : h_{OH,i} ,\ g^{o}_{OH,i} \right) } 
\end{equation}
\begin{equation}
\I \label{EES Eqn:45}
\K{call} \F{enthalpygibbs}{ \left( \SC{H2},\ T_{final,i} : h_{H2,i} ,\ g^{o}_{H2,i} \right) } 
\end{equation}
\begin{equation}
\I \label{EES Eqn:46}
\K{call} \F{enthalpygibbs}{ \left( \SC{CO},\ T_{final,i} : h_{CO,i} ,\ g^{o}_{CO,i} \right) } 
\end{equation}

\vspace{0.04in}
\noindent
\rm Call EnthalpyGibbs('CH4',T$_{final,i}$ : h$_{CH4,i}$ , g$^{o}$$_{CH4,i}$)\newline
	Call EnthalpyGibbs('C2H6',T$_{final,i}$ : h$_{C2H6,i}$ , g$^{o}$$_{C2H6,i}$)\newline
	h$_{fuel,i}$ = 0.9 * h$_{CH4,i}$ + 0.1 * h$_{C2H6,i}$\newline
	g$^{o}$$_{fuel,i}$ = 0.9 * g$^{o}$$_{CH4,i}$ + 0.1 * g$^{o}$$_{C2H6,i}$

\vspace{0.04in}
\noindent
\rm Standard-state Gibbs Free Energy change for our 6 reactions.

\vspace{0.04in}
\noindent
\rm This block of code was replaced with the manual calculation of the equilibrium constants present at appendix A of this file. I have no Idea why this segment didnt work, especially seeing 	that it performed just fine for the first and third segments and just had trouble calculating part 2 (it gave me the rich value of the AF ratio corresponding to 1600 [K] final temprature, one 	that claimed a[2] = 1.008 and $\dot{m}$$_{2}$ $<$ 0), but since I didnt have time to troubleshoot it proparly, I just used the brute force method of calculating with pen and paper

\vspace{0.04in}
\noindent
\rm ${\Delta G$}$^{o}$$_{1,i}$ = (2 * g$^{o}$$_{N,i}$)      -   g$^{o}$$_{N2,i}$\newline
	DELTAG$^{o}$$_{2,i}$ = (2 * g$^{o}$$_{O,i}$)      -   g$^{o}$$_{O2,i}$\newline
	DELTAG$^{o}$$_{3,i}$ = (2 * g$^{o}$$_{H,i}$)      -   g$^{o}$$_{H2,i}$\newline
	DELTAG$^{o}$$_{4,i}$ = (2 * g$^{o}$$_{OH,i}$)    -   g$^{o}$$_{O2,i}$   -   g$^{o}$$_{H2,i}$\newline
	DELTAG$^{o}$$_{5,i}$ = (2 * g$^{o}$$_{CO2,i}$)  -   g$^{o}$$_{O2,i}$   -   (2 * g$^{o}$$_{CO,i}$)\newline
	DELTAG$^{o}$$_{6,i}$ = (4 * g$^{o}$$_{OH,i}$)    -   g$^{o}$$_{O2,i}$   -   (2 * g$^{o}$$_{H2O,i}$)

\vspace{0.04in}
\noindent
\rm Law of Mass Action for reactions 1 through 6

\vspace{0.04in}
\noindent
\rm This block of code was replaced with the manual calculation of the equilibrium constants present at appendix A of this file. I have no Idea why this segment didnt work, especially seeing 	that it performed just fine for the first and third segments and just had trouble calculating part 2 (it gave me the rich value of the AF ratio corresponding to 1600 [K] final temprature, one 	that claimed a[2] = 1.008 and $\dot{m}$$_{2}$ $<$ 0), but since I didnt have time to troubleshoot it proparly, I just used the brute force method of calculating with pen and paper

\vspace{0.04in}
\noindent
\rm ${\Delta G$}$^{o}$$_{1,i}$ = -1 * R * T$_{final,i}$ * ln(K$_{1,i}$)\newline
	DELTAG$^{o}$$_{2,i}$ = -1 * R * T$_{final,i}$ * ln(K$_{2,i}$)\newline
	DELTAG$^{o}$$_{3,i}$ = -1 * R * T$_{final,i}$ * ln(K$_{3,i}$)\newline
	DELTAG$^{o}$$_{4,i}$ = -1 * R * T$_{final,i}$ * ln(K$_{4,i}$)\newline
	DELTAG$^{o}$$_{5,i}$ = -1 * R * T$_{final,i}$ * ln(K$_{5,i}$)\newline
	DELTAG$^{o}$$_{6,i}$ = -1 * R * T$_{final,i}$ * ln(K$_{6,i}$)

\vspace{0.04in}
\noindent
\rm Definition of equilibrium constant for reactions 1 through 6
\begin{equation}
\I \label{EES Eqn:47}
K_{1,i}   \cdot    y_{N2,i	}=        \left( y_{N,i}^{2} \right)    \cdot     \left( P/P_{ref} \right)  
\end{equation}
\begin{equation}
\I \label{EES Eqn:48}
K_{2,i}   \cdot    y_{O2,i	}=        \left( y_{O,i}^{2} \right)    \cdot     \left( P/P_{ref} \right)  
\end{equation}
\begin{equation}
\I \label{EES Eqn:49}
K_{3,i}   \cdot    y_{H2,i	}=        \left( y_{H,i}^{2} \right)    \cdot     \left( P/P_{ref} \right)  
\end{equation}
\begin{equation}
\I \label{EES Eqn:50}
K_{4,i}   \cdot    y_{O2,i}   \cdot    y_{H2,i	}=        \left( y_{OH,i}^{2} \right)  
\end{equation}
\begin{equation}
\I \label{EES Eqn:51}
K_{5,i}   \cdot    y_{O2,i}   \cdot     \left( y_{CO,i}^{2} \right)    \cdot     \left( P/P_{ref} \right) 	=        \left( y_{CO2,i}^{2} \right)  
\end{equation}
\begin{equation}
\I \label{EES Eqn:52}
K_{6,i}   \cdot    y_{O2,i}   \cdot     \left( y_{H2O,i}^{2} \right) 	=        \left( y_{OH,i}^{4} \right)    \cdot     \left( P/P_{ref} \right)  
\end{equation}

\vspace{0.04in}
\noindent
\rm Find the enthalpy of the reactants
\begin{equation}
\I \label{EES Eqn:53}
\V{HR} _{i} =  h_{fuel,inlet} + a_{i} \cdot  h_{O2,inlet} + 3.76 \cdot  a_{i} \cdot  h_{N2,inlet} 
\end{equation}

\vspace{0.04in}
\noindent
\rm Find the enthalpy of products
\begin{equation}
\I \label{EES Eqn:54}
\V{HP} _{i} =  \left( b_{i} \cdot  h_{CO2,i} \right)  +  \left( c_{i} \cdot  h_{H2O,i} \right)  +  \left( d_{i} \cdot  h_{N2,i} \right)  +  \left( e_{i} \cdot  h_{O2,i} \right)  +  \left( f_{i} \cdot  h_{O,i} \right)  +  \left( g_{i} \cdot  h_{N,i} \right)  +  \left( j_{i} \cdot  h_{H,i} \right)  +  \left( k_{i} \cdot  h_{OH,i} \right)  +  \left( m_{i} \cdot  h_{H2,i} \right)  +  \left( n_{i} \cdot  h_{CO,i} \right)  
\end{equation}

\vspace{0.04in}
\noindent
\rm Apply an adiabatic energy balance to determine the product temperature
\begin{equation}
\I \label{EES Eqn:55}
\V{HR} _{i} = \V{HP} _{i} 
\end{equation}
\begin{equation}
\label{EES Eqn:56}
\K{end} 
\end{equation}

\vspace{0.04in}
\noindent
\rm 1 kmol of fuel weighs the same as 0.9 kmol of CH4 and 0.1 kmol of C2H6, = 0.9 * 16.043 + 0.1 * 30.07 = 17.4457 kg/kmol, so 0.07 kg/s fuel equals to 0.0040124501 kmol/s of fuel.\newline
and as 1 kmol of air weighs 28.97 kg/kmol, then a kmols of air per 1 kmol of fuels equals [0.0040124501*28.97]*a = 0.1162406782*a kg/s of air for 0.07 kg/s fuel
\begin{equation}
\label{EES Eqn:57}
\dot {m}_{1} =  \left( 1 + 3.76 \right)  \cdot  a_{1} \cdot  0.1162406782   \   \left[ \rm kg/s \right] 
\end{equation}
\rm
\begin{equation}
\label{EES Eqn:58}
\dot {m}_{1} + \dot {m}_{2} =  \left( 1 + 3.76 \right)  \cdot  a_{2} \cdot  0.1162406782   \   \left[ \rm kg/s \right] 
\end{equation}
\rm
\begin{equation}
\label{EES Eqn:59}
\dot {m}_{1} + \dot {m}_{2} + \dot {m}_{3} =  \left( 1 + 3.76 \right)  \cdot  a_{3} \cdot  0.1162406782   \   \left[ \rm kg/s \right] 
\end{equation}
\rm

\vspace{0.04in}
\noindent
\rm ~~~~~~~~~~Appendix A~~~~~~~~~~

\vspace{0.04in}
\noindent
\rm From Table 1.7a we can give estimated initial values of K to our program:\newline
K$_{1}$ = K$_{P,5}$ $^{}$ 2      /      K$_{2}$ = K$_{P,1}$ $^{}$ 2      /      K$_{3}$ = K$_{P,2}$ $^{}$ 2      /      K$_{4}$ = K$_{P,3}$ $^{}$ 2      /      K$_{5}$ = (K$_{P,10}$ / K$_{P,9}$) $^{}$ 2      /      K$_{6}$ = (K$_{P,3}$ / sqrt(K$_{P,4}$)) $^{}$ 4
\begin{equation}
\label{EES Eqn:60}
K_{1,1} = 3.076\times 10^{-23} 
\end{equation}
\begin{equation}
\label{EES Eqn:61}
K_{1,2} = 4.467\times 10^{-25} 
\end{equation}
\begin{equation}
\label{EES Eqn:62}
K_{1,3} = 2.779\times 10^{-32} 
\end{equation}
\begin{equation}
\label{EES Eqn:63}
K_{2,1} = 2.000\times 10^{-9} 
\end{equation}
\begin{equation}
\label{EES Eqn:64}
K_{2,2} = 2.099\times 10^{-10} 
\end{equation}
\begin{equation}
\label{EES Eqn:65}
K_{2,3} = 3.133\times 10^{-14} 
\end{equation}
\begin{equation}
\label{EES Eqn:66}
K_{3,1} = 2.177\times 10^{-8} 
\end{equation}
\begin{equation}
\label{EES Eqn:67}
K_{3,2} = 2.965\times 10^{-9} 
\end{equation}
\begin{equation}
\label{EES Eqn:68}
K_{3,3} = 1.225\times 10^{-12} 
\end{equation}
\begin{equation}
\label{EES Eqn:69}
K_{4,1} = 0.1514 
\end{equation}
\begin{equation}
\label{EES Eqn:70}
K_{4,2} = 0.1086 
\end{equation}
\begin{equation}
\label{EES Eqn:71}
K_{4,3} = 0.02965 
\end{equation}
\begin{equation}
\label{EES Eqn:72}
K_{5,1} = 2.138\times 10^{8} 
\end{equation}
\begin{equation}
\label{EES Eqn:73}
K_{5,2} = 2.547\times 10^{9} 
\end{equation}
\begin{equation}
\label{EES Eqn:74}
K_{5,3} = 4.325\times 10^{13} 
\end{equation}
\begin{equation}
\label{EES Eqn:75}
K_{6,1} = 9.204\times 10^{-12} 
\end{equation}
\begin{equation}
\label{EES Eqn:76}
K_{6,2} = 5.176\times 10^{-13} 
\end{equation}
\begin{equation}
\label{EES Eqn:77}
K_{6,3} = 6.576\times 10^{-18} 
\end{equation}

\end{document}
