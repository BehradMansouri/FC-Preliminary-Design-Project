%LaTeX reported generated by EES
\documentclass[10pt,fleqn]{article}
\mathindent 0.0in
\usepackage[ansinew]{inputenc}
\usepackage{times}
\usepackage{graphicx}
\usepackage{color}
\usepackage{textcomp}
\definecolor{silver}{rgb}{0.75,0.75,0.75}
\definecolor{gray}{rgb}{0.5,0.5,0.5}
\definecolor{aqua}{rgb}{0.5,1,1}
\definecolor{navy}{rgb}{0.0,0.0,0.5}
\definecolor{orange}{rgb}{1.0,0.5,0.0}
\definecolor{teal}{rgb}{0.25,0.5,0.5}
\definecolor{olive}{rgb}{0.5,0.5,0.0}
\definecolor{purple}{rgb}{0.5,0.0,0.5}
\definecolor{brown}{rgb}{0.5,0.25,0.0}
\definecolor{fuchsia}{rgb}{1.0,0.5,1.0}
\definecolor{buff}{rgb}{1.0,0.94,0.80}
\definecolor{lime}{rgb}{0.5,1.0,0.0}
\setlength{\headsep}{-0.6in}
\setlength{\textheight}{9in}
\setlength{\footskip}{1.0 in}
\setlength{\oddsidemargin}{-0.2in}
\setlength{\evensidemargin}{-0.2in}
\setlength{\textwidth}{6.8in}
\usepackage{longtable}
\def\headline#1{\hbox to \hsize{\hrulefill\quad\lower .3em\hbox{#1}\quad\hrulefill}}
\newcommand{\abs}[1]{\left|#1\right|}
\newcommand{\F}[1]{\mbox{$#1$}}
\newcommand{\K}[1]{\mbox{\sf#1\ \ \mit}}
\newcommand{\KS}[1]{\mbox{\sf\ \ #1\ \ \mit}}
\newcommand{\SC}[1]{\mbox{`#1'}\  }
\newcommand{\V}[1]{\mbox{$ #1 $}}
\newcommand{\I}{\mbox{\hspace{0.20in}}}
\newcommand{\temperature}{\mathrm{T}}
\newcommand{\pressure}{\mathrm{P}}
\newcommand{\volume}{\mathrm{v}}
\newcommand{\density}{\mathrm{\rho}}
\newcommand{\intenergy}{\mathrm{u}}
\newcommand{\enthalpy}{\mathrm{h}}
\newcommand{\entropy}{\mathrm{s}}
\newcommand{\molarmass}{\mathrm{MW}}
\newcommand{\enthalpyfusion}{\mathrm{\Delta h_{fusion}}}
\newcommand{\quality}{\mathrm{x}}
\newcommand{\viscosity}{\mathrm{\mu}}
\newcommand{\conductivity}{\mathrm{k}}
\newcommand{\prandtl}{\mathrm{P_r}}
\newcommand{\cp}{\mathrm{c_p}}
\newcommand{\cv}{\mathrm{c_v}}
\newcommand{\specheat}{\mathrm{c_p}}
\newcommand{\soundspeed}{\mathrm{c}}
\newcommand{\wetbulb}{\mathrm{wb}}
\newcommand{\humrat}{\mathrm{\omega}}
\newcommand{\acentricfactor}{\mathrm{\omega}}
\newcommand{\relhum}{\mathrm{\phi}}
\newcommand{\dewpoint}{\mathrm{DP}}
\newcommand{\volexpcoef}{\mathrm{\beta}}
\newcommand{\compressibilityfactor}{\mathrm{Z}}
\newcommand{\surfacetension}{\mathrm{\gamma}}
\newcommand{\tcrit}{\mathrm{T_{crit}}}
\newcommand{\pcrit}{\mathrm{P_{crit}}}
\newcommand{\vcrit}{\mathrm{v_{crit}}}
\newcommand{\ttriple}{\mathrm{T_{triple}}}
\newcommand{\fugacity}{\mathrm{fugacity}}
\newcommand{\tsat}{\mathrm{T_{sat}}}
\newcommand{\psat}{\mathrm{P_{sat}}}
\newcommand{\eklj}{\mathrm{ek_{LJ}}}
\newcommand{\sigmalj}{\mathrm{\sigma_{LJ}}}
\newcommand{\isentropicexponent}{\mathrm{k_{s}}}
\newcommand{\thermaldiffusivity}{\mathrm{\alpha}}
\newcommand{\kinematicviscosity}{\mathrm{\nu}}
\newcommand{\isothermalcompress}{\mathrm{K_{T}}}
\newcommand{\henryconstantwater}{\mathrm{HenryConst}}
\begin{document}
\begin{center}
\bf \mbox{PART 1 - A}
\vspace{0.2 in}
\end{center}
\subsection*{Equations}
\begin{equation}
\label{EES Eqn:1}
\K{function} \F{getenthalpy}{ \left( S\$,\ T \right) } 
\end{equation}

\vspace{0.04in}
\noindent
\rm This function uses the NASA external procedure to return the enthalpy of species S\$ at T.
\begin{equation}
\I \label{EES Eqn:2}
\K{call} \F{NASA}{ \left( S\$,\ T:\V{cp} ,\ \V{geTenthalpy} ,\ s \right) } 
\end{equation}
\begin{equation}
\label{EES Eqn:3}
\K{end} \V{getenthalpy}  
\end{equation}

\vspace{0.04in}
\noindent
\vspace{0.1 in}
\rm Considering that for this segment, the equilibrium equations are not given to us, we assume the only tool in our disposal for figuring out the mole fractions in the products in conservation of mass.\newline
 \newline
0.9 CH4  +  a$_{1}$ (O2 + 3.76 N2)   $\leftarrow$---$\rightarrow$  b$_{1}$ CO2  +  c$_{1}$ H2O  +  d$_{1}$ N2  +  e$_{1}$ O2  +  f$_{1}$ C2H6\newline
0.1 C2H6  +  a$_{2}$ (O2 + 3.76 N2)   $\leftarrow$---$\rightarrow$  b$_{2}$ CO2  +  c$_{2}$ H2O  +  d$_{2}$ N2  +  e$_{2}$ O2  +  f$_{2}$ C2H6\newline
this form of the equation too is sufficiently good enough if we use HR$_{1}$ + HR$_{2}$ = HP$_{1}$ + HP$_{2}$, but if we change our energy equation to HR$_{1}$ = HP$_{1}$ \& HR$_{2}$ = HP$_{2}$ meaning we assume that the two combustion reactions are completely isolated from eachother (there is no heat exchange between them). here we are finally able to solve for our mole fractions only after removing O2 from out products. I did not opt for this solution however as it seemed to make one asumption too many\newline
Thus we arrive to the solution I found the most satisfying:\newline
 \newline
0.9 CH4  + 0.1 C2H6 +  a (O2 + 3.76 N2)   $\leftarrow$---$\rightarrow$  b CO2  +  c H2O  +  d N2  +  e O2  +  $>$$>$$>$f CH4  +  g C2H6$<$$<$$<$\newline
OR\newline
C1.1H4.2  +  a (O2 + 3.76 N2)   $\leftarrow$---$\rightarrow$  b CO2  +  c H2O  +  d N2  +  e O2  +  $>$$>$$>$f C1.1H4.2$<$$<$$<$\newline
by combining the CH4 and the C2H6 fuels into a hybrid C1.1H4.2 (just as how we did it for natual gas) our equation is finally solveable. the only downside is that we cannot determine that in our unbernt fuel what percentage is CH4 and what amount of it belongs to C2H6, but since our problem only is only concerend with input air, I found this to be an acceptable compromise
\begin{equation}
\label{EES Eqn:4}
T_{air} = 460   \   \left[ \rm K \right] 
\end{equation}
\rm
\begin{equation}
\label{EES Eqn:5}
T_{fuel} = 430   \   \left[ \rm K \right] 
\end{equation}
\rm
\begin{equation}
\label{EES Eqn:6}
T_{final,1}= 1700   \   \left[ \rm K \right] 
\end{equation}
\rm
\begin{equation}
\label{EES Eqn:7}
T_{final,2}= 1600   \   \left[ \rm K \right] 
\end{equation}
\rm
\begin{equation}
\label{EES Eqn:8}
T_{final,3}= 1300   \   \left[ \rm K \right] 
\end{equation}
\rm
\begin{equation}
\label{EES Eqn:9}
h_{N2,inlet} = \F{getenthalpy}{ \left( \SC{N2},\ T_{air} \right) } 
\end{equation}
\begin{equation}
\label{EES Eqn:10}
h_{O2,inlet} = \F{getenthalpy}{ \left( \SC{O2},\ T_{air} \right) } 
\end{equation}
\begin{equation}
\label{EES Eqn:11}
h_{CH4,inlet} = \F{getenthalpy}{ \left( \SC{CH4},\ T_{fuel} \right) } 
\end{equation}
\begin{equation}
\label{EES Eqn:12}
h_{C2H6,inlet} = \F{getenthalpy}{ \left( \SC{C2H6},\ T_{fuel} \right) } 
\end{equation}
\begin{equation}
\label{EES Eqn:13}
h_{fuel,inlet} = 0.9 \cdot  h_{CH4,inlet} + 0.1 \cdot  h_{C2H6,inlet} 
\end{equation}

\vspace{0.04in}
\noindent
\rm Stoichiometry for a basis of 1 kmol of fuel
\begin{equation}
\label{EES Eqn:14}
1.1 = \V{b	}  
\mbox{\I Carbon balance}
\end{equation}
\begin{equation}
\label{EES Eqn:15}
4.2 =  2 \cdot  \V{c	}  
\mbox{\I Hydrogen balance}
\end{equation}
\begin{equation}
\label{EES Eqn:16}
\K{duplicate} i=1,\ 3 
\end{equation}

\vspace{0.04in}
\noindent
\rm Stoichiometry for a basis of 1 kmol of fuel - Continued
\begin{equation}
\I \label{EES Eqn:17}
2 \cdot  a_{i} = 2 \cdot  b + c + 2 \cdot  e_{i}	 
\mbox{\I Oxygen balance}
\end{equation}
\begin{equation}
\I \label{EES Eqn:18}
3.76 \cdot  2 \cdot  a_{i} = 2 \cdot  d_{i}	 
\mbox{\I Nitrogen balance}
\end{equation}

\vspace{0.04in}
\noindent
\rm The following equations provide the enthalpy for each chemical species at the inlet tempretures and T$_{final}$ and the reference pressure of 10 bar. The NASA external procedure is used in the Function 	getenthalpy to calculate h at the equilibrium temperature, which is determined from an energy balance.
\begin{equation}
\I \label{EES Eqn:19}
h_{CO2,i} = \F{getenthalpy}{ \left( \SC{CO2},\ T_{final,i} \right) } 
\end{equation}
\begin{equation}
\I \label{EES Eqn:20}
h_{H2O,i} = \F{getenthalpy}{ \left( \SC{H2O},\ T_{final,i} \right) } 
\end{equation}
\begin{equation}
\I \label{EES Eqn:21}
h_{N2,i} = \F{getenthalpy}{ \left( \SC{N2},\ T_{final,i} \right) } 
\end{equation}
\begin{equation}
\I \label{EES Eqn:22}
h_{O2,i} = \F{getenthalpy}{ \left( \SC{O2},\ T_{final,i} \right) } 
\end{equation}

\vspace{0.04in}
\noindent
\rm h$_{CH4,i}$ = getenthalpy('CH4',T$_{final,i}$)\newline
	h$_{C2H6,i}$ = getenthalpy('C2H6',T$_{final,i}$)\newline
	h$_{fuel,i}$ = 0.9 * h$_{CH4,i}$ + 0.1 * h$_{C2H6,i	only}$ truly accurate if our fuels burn at a rate proportional to their initial molar fraction, but is servisable enough as an estimate

\vspace{0.04in}
\noindent
\rm Find the enthalpy of the reactants
\begin{equation}
\I \label{EES Eqn:23}
\V{HR} _{i} =  h_{fuel,inlet} + a_{i} \cdot  h_{O2,inlet} + 3.76 \cdot  a_{i} \cdot  h_{N2,inlet} 
\end{equation}

\vspace{0.04in}
\noindent
\rm Find the enthalpy of products
\begin{equation}
\I \label{EES Eqn:24}
\V{HP} _{i} = b \cdot  h_{CO2,i} + c \cdot  h_{H2O,i} + d_{i} \cdot  h_{N2,i} + e_{i} \cdot  h_{O2,i} 
\end{equation}

\vspace{0.04in}
\noindent
\rm Apply an adiabatic energy balance to determine the product temperature
\begin{equation}
\I \label{EES Eqn:25}
\V{HR} _{i} = \V{HP} _{i} 
\end{equation}
\begin{equation}
\label{EES Eqn:26}
\K{end} 
\end{equation}

\vspace{0.04in}
\noindent
\rm 1 kmol of fuel weighs the same as 0.9 kmol of CH4 and 0.1 kmol of C2H6, = 0.9 * 16.043 + 0.1 * 30.07 = 17.4457 kg/kmol, so 0.07 kg/s fuel equals to 0.0040124501 kmol/s of fuel.\newline
and as 1 kmol of air weighs 28.97 kg/kmol, then a kmols of air per 1 kmol of fuels equals [0.0040124501*28.97]*a = 0.1162406782*a kg/s of air for 0.07 kg/s fuel
\begin{equation}
\label{EES Eqn:27}
\dot {m}_{1} =  \left( 1 + 3.76 \right)  \cdot  a_{1} \cdot  0.1162406782   \   \left[ \rm kg/s \right] 
\end{equation}
\rm
\begin{equation}
\label{EES Eqn:28}
\dot {m}_{1} + \dot {m}_{2} =  \left( 1 + 3.76 \right)  \cdot  a_{2} \cdot  0.1162406782   \   \left[ \rm kg/s \right] 
\end{equation}
\rm
\begin{equation}
\label{EES Eqn:29}
\dot {m}_{1} + \dot {m}_{2} + \dot {m}_{3} =  \left( 1 + 3.76 \right)  \cdot  a_{3} \cdot  0.1162406782   \   \left[ \rm kg/s \right] 
\end{equation}
\rm

\end{document}
